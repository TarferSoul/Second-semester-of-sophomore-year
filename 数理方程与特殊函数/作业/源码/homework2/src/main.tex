%! Author = Carl
%! Date = 2024/3/3

% Preamble
\documentclass[11pt]{article}

% Packages
\usepackage[T1]{fontenc}% optional T1 font encoding
\usepackage{graphicx}
\usepackage{color}
\usepackage{cite}
%\usepackage{tgpagella}
\usepackage{libertine}
\usepackage{subfigure}
\usepackage{amsmath}
\usepackage{amsthm}
\usepackage{ctex}
\usepackage{geometry}
% Document
\geometry{a4paper,left=2cm,right=2cm,top=1cm,bottom=3cm}

\begin{document}
\title{\vspace{-2cm}HOMEWORK\\ 数理方程与特殊函数}
\author{王翎羽\quad U202213806\quad 提高2201班}
\maketitle

\section*{练习三}
\begin{enumerate}
    \item 求下列固有值问题的固有值和固有函数:
    \begin{equation*}
            \left\{
             \begin{array}{lr}
             X^{''}(x)+\lambda X(x)=0, &  \\
             X(0)=X^{'}(l)=0, &
             \end{array}
            \right.
            \end{equation*}
    解:分三种情况讨论$\lambda$.
    \begin{enumerate}
        \item[(1)] 当$\lambda <0$时,该问题没有平凡解。方程的通解为$X(x)=Ae^{\sqrt {-\lambda }x}+Be^{-\sqrt {-\lambda }x}$.\\将边界条件代入,得到:$A+B=0$和$A\sqrt {-\lambda }e^{\sqrt {-\lambda }l}+B\sqrt {-\lambda }e^{-\sqrt {-\lambda }l}=0$.\\所以有$A=B=0$,方程没有非平凡解.
        \item[(2)] 当$\lambda =0$时,该问题没有平凡解。方程的通解为$X(x)=(A+Bx)e^{\lambda x}$,很显然,将边界条件代入,得到:$A+B=0$,方程没有非平凡解.
        \item[(3)] 当$\lambda >0$时,方程的通解为$X(x)=A\cos(\sqrt {\lambda }x)+B\sin(\sqrt {\lambda }x)$,代入方程的边界条件得到:\\ $-A\sqrt {\lambda }\sin(\sqrt {\lambda }l)+B\sqrt {\lambda }\cos(\sqrt {\lambda }l)=0$和$A=0$.即$B\cos(\sqrt {\lambda }l)=0$,解得$\lambda =\lambda_n=(\frac{(2n+1)\pi}{2l})^2(n=0,1,2,\dots)$.$X_n(x)=B_{n}\sin(\frac{(2n+1)\pi}{2l}))(n=0,1,2,\dots)$

    \end{enumerate}

    \item 求如下定解问题的解:
    \begin{equation*}
            \left\{
             \begin{array}{lr}
             u_{tt}=a^2u_{xx},\quad 0<x<l,\quad t>0 &  \\
             u(0,t)=u_x(l,t)=0, & \\
             u(x,0)=3\sin(\frac{3\pi x}{2l})+6\sin(\frac{5\pi x}{2l}), &    \\
             u_t(x,0)=0. &
             \end{array}
            \right.
            \end{equation*}
    解:令$u(x,t)=X(x)T(t)$,代入方程分离变量得到两个ODE,\\$T^{''}+a^{2}\lambda T=0$和$X^{''}+\lambda X=0$,由边界条件可知:$X(0)=X^{'}(l)=0$.\\求这个ODE的非平凡解:已知当$\lambda>0$时,问题才会有非平凡解.\\通解为:$X(x)=A\cos(\sqrt {\lambda }x)+B\sin(\sqrt {\lambda}x)$.\\代入边界条件得到:$A=0$和$B\sqrt {\lambda}\cos(\sqrt {\lambda}x)=0(n=1,2,\dots)$.所以$\lambda = \lambda_n=(\frac{(2n+1)\pi}{2l})^2$.\\从而找到一族非零解:$X_n=B_n\sin(\frac{(2n+1)\pi x}{2l})(n=1,2,\dots)$.
    \\ 对于ODE:$T^{''}+a^{2}\lambda T=0$ 而言,将特征值$\lambda$代入方程得到:\\$T^{''}(t)+(\frac{(2n+1)\pi a}{2l})^{2}T(t)=0,$通解为:$T_n(t)=C_n\cos(\frac{(2n+1)\pi at}{2l})+D_n\sin(\frac{(2n+1)\pi at}{2l})(n=0,1,2,\dots)$.\\由叠加定理得原问题的解表示为:$u(x,t)=\sum_{n=0}^{\infty}\big[a_n\cos\frac{(2n+1)\pi at}{2l}+b_n\sin\frac{(2n+1)\pi at}{2l}]\sin\frac{(2n+1)\pi x}{2l}$,\\其中$a_n=B_{n}C_n$,\quad $b_n=B_nD_n$是任意常数.\\结合初值条件可得:
    \begin{equation*}
            \left\{
             \begin{array}{lr}
             \sum_{n=0}^{\infty}a_n\sin\frac{(2n+1)\pi x}{2l}=3\sin(\frac{3\pi x}{2l})+6\sin(\frac{5\pi x}{2l}), &  \\
             \sum_{n=0}^{\infty}b_n\frac{(2n+1)\pi a}{2l}\sin\frac{(2n+1)\pi x}{2l}=0 & \\
             \end{array}
            \right.
            \end{equation*}
    得到\begin{equation*}
            \left\{
             \begin{array}{lr}
             a_n=\frac{2}{l}\int_0^l(3\sin(\frac{3\pi x}{2l})+6\sin(\frac{5\pi x}{2l}))\sin\frac{(2n+1)\pi x}{2l}dx, &  \\
             b_n=0. & \\
             \end{array}
            \right.
            \end{equation*}
    解得\begin{equation*}
            a_n=\begin{cases}
                    3,& n=1\\
                    6.&n=2
            \end{cases}
            \end{equation*}
    所以$u(x,t)=3\cos\frac{3\pi at}{2l}\sin\frac{3\pi x}{2l}+6\cos\frac{5\pi at}{2l}\sin\frac{5\pi x}{2l}$


    \item 求解以下定解问题:\\
    \begin{equation*}
            \left\{
             \begin{array}{lr}
             u_{tt}=u_{xx}+2u,\quad 0<x<1,\quad t>0 &  \\
             u(0,t)=u_x(1,t)=0, & \\
             u(x,0)=0, &    \\
             u_t(x,0)=\sqrt {\pi^2-2}\sin\pi x. &
             \end{array}
            \right.
            \end{equation*}
    解: 令$u(x,t)=X(x)T(t)$,代入方程分离变量得到两个ODE,$T^{''}+(\lambda-2) T=0$和$X^{''}+\lambda X=0$.\\由边界条件可知:$X(0)=X(1)=0$.\\求后者的非平凡解:已知当$\lambda>0$时,问题才会有非平凡解.\\通解为:$X(x)=A\cos(\sqrt{\lambda}x)+B\sin(\sqrt {\lambda}x)$.由边界条件得:$A=0$和$A\cos(\sqrt {\lambda})+B\sin(\sqrt {\lambda})=0$.\\即$\lambda=\lambda_n=(n\pi)^2$,从而找到一族非零解:$X(x)=B_n\sin(n\pi x)$.
        \\对于ODE:$T^{''}+(\lambda-2) T=0$而言,将特征值$\lambda$代入,得到:$T^{''}(t)-((n\pi)^2-2)T(t)=0$.\\其通解为:$T_n(t)=C_{n}\cos\sqrt {n^2\pi^2-2}t+D_n\sin\sqrt {n^2\pi^2-2}t$.\\原问题的解可表示为:$u(x,t)=\sum_{n=0}^{\infty}\big[a_{n}\cos\sqrt {n^2\pi^2-2}t+b_n\sin\sqrt {n^2\pi^2-2}t]\sin(n\pi x)$.\\其中$a_n=B_{n}C_n$,\quad $b_n=B_nD_n$是任意常数.\\结合初值条件可得:\begin{equation*}
            \left\{
             \begin{array}{lr}
             \sum_{n=0}^{\infty}a_{n}\sin (n\pi x)=0, &  \\
             \sum_{n=0}^{\infty}b_n\sqrt {n^2\pi^2-2}\sin (n\pi x)=\sqrt {\pi^2-2}\sin(\pi x). & \\
             \end{array}
            \right.
            \end{equation*}
    得到\begin{equation*}
            \left\{
             \begin{array}{lr}
             a_n=0, &  \\
             b_n\sqrt {n^2\pi^2-2}=2\int_0^1\sqrt {\pi^2-2}\sin(\pi x)\sin (n\pi x)dx. & \\
             \end{array}
            \right.
            \end{equation*}
    解得\begin{equation*}
            b_n=\begin{cases}
                    1,& n=1\\
                    0.&n\neq 1
            \end{cases}
            \end{equation*}
    所以$u(x,t)=\sin\sqrt {\pi^2-2}t\sin(\pi x).$

\end{enumerate}
\setlength{\topmargin}{-18mm}
\section*{练习四}
    \begin{enumerate}
        \item 求如下定解问题的解:
    \begin{equation*}
            \left\{
             \begin{array}{lr}
             u_{t}=a^2u_{xx},\quad 0<x<l,\quad t>0 &  \\
             u(0,t)=u(l,t)=0, & \\
             u(x,0)=x(l-x), &    \\
             \end{array}
            \right.
            \end{equation*}
        解:令$u(x,t)=X(x)T(t)$,代入方程分离变量得到两个ODE,$X^{''}+\lambda X=0$和$T^{'}+a^2\lambda T=0$.\\由边界条件可知:$X(0)=X(l)=0$.求前者的非平凡解:已知当$\lambda>0$时,问题才会有非平凡解.\\通解为:$X(x)=A\cos(\sqrt {\lambda}x)+B\sin(\sqrt {\lambda}x)$.将边界条件代入,得到:\\$A=0$和$B\sin(\sqrt {\lambda}x)=0(n=1,2,\dots)$.所以$\lambda = \lambda_n=(\frac{n\pi}{l})^2$.\\从而找到一族非零解:$X_n=B_n\sin(\frac{n\pi x}{l})(n=1,2,\dots)$.
            \\对于ODE:$T^{'}+a^2\lambda T=0$而言,将特征值$\lambda$代入,得到:$T^{'}+a^2(\frac{n\pi}{l})^2 T=0$.其通解为:$T(t)=Ce^{-a^2\lambda t}$.\\原问题的解可表示为:$u(x,t)=\sum_{n=0}^{\infty}a_ne^{-a^2(\frac{n\pi}{l})^2 t}\sin(\frac{n\pi x}{l})$.其中$a_n=B_{n}C_n$为任意常数.\\由初值条件,得到$\sum_{n=0}^{\infty}a_n\sin(\frac{n\pi x}{l})=x(l-x).$\\ 所以$a_n=\infty^l_0 x(l-x)\sin(\frac{n\pi x}{l})dx=(-\frac{2l^2}{n\pi}+\frac{2x^2}{n\pi})-\frac{4l^2}{n^3\pi^3}\cos(n\pi)+\frac{4l^2}{n^3\pi^3}$.
            解得\begin{equation*}
            a_n=\begin{cases}
                    \frac{2(x^2-l^2)}{n\pi},& n\quad is\quad even\\
                    \frac{8l^2}{n^3\pi^3}+\frac{2(l^2-x^2)}{n\pi}.& n\quad is \quad odd
            \end{cases}
            \end{equation*}
        所以$u(x,t)=\sum_{n=1}^{\infty}\frac{4l^2}{n^3\pi^3}(1-(-1)^{n})\sin(\frac{n\pi x}{l})e^{-(\frac{n\pi x}{l})^2t}$


        \item 求下列固有值问题的固有值和固有函数:
    \begin{equation*}
            \left\{
             \begin{array}{lr}
             X^{''}(x)+\lambda X(x)=0, &  \\
             X^{'}(0)=X(l)=0, &
             \end{array}
            \right.
            \end{equation*}
    解:分三种情况讨论$\lambda$.
    \begin{enumerate}
        \item[(1)] 当$\lambda <0$时,该问题没有平凡解。方程的通解为$X(x)=Ae^{\sqrt {-\lambda }x}+Be^{-\sqrt {-\lambda }x}$.\\将边界条件代入,得到:$A+B=0$和$Ae^{\sqrt {-\lambda }l}+Be^{-\sqrt {-\lambda }l}=0$.\\所以有$A=B=0$,方程没有非平凡解.
        \item[(2)] 当$\lambda =0$时,该问题没有平凡解。方程的通解为$X(x)=(A+Bx)e^{\lambda x}$,很显然,将边界条件代入,得到:$A+B=0$,方程没有非平凡解.
        \item[(3)] 当$\lambda >0$时,方程的通解为$X(x)=A\cos(\sqrt {\lambda }x)+B\sin(\sqrt {\lambda }x)$,代入方程的边界条件得到:\\ $B\cos(\sqrt {\lambda l})=0$和$A=0$.解得$\lambda =\lambda_n=(\frac{(2n+1)\pi}{2l})^2(n=1,2,\dots)$.$X_n(x)=B_{n}\cos(\frac{(2n+1)\pi x}{2l})(n=1,2,\dots)$

    \end{enumerate}

        \item 求如下定解问题的解:
    \begin{equation*}
            \left\{
             \begin{array}{lr}
             u_{t}=u_{xx},\quad 0<x<2,\quad t>0 &  \\
             u_x(0,t)=u(2,t)=0, & \\
             u(x,0)=4\cos\frac{5\pi x}{4}, &    \\
             \end{array}
            \right.
            \end{equation*}
        解:令$u(x,t)=X(x)T(t)$,代入方程分离变量得到两个ODE,$X^{''}+\lambda X=0$和$T^{'}+\lambda T=0$.\\由前题易得:$X(x)=B_n\cos\big(\frac{(2n+1)\pi x}{4}\big)$.代入第二个ODE得:$T_n=C_{n}e^{-(\frac{\pi+2n\pi}{4})^2 t}$.\\ $u(x,t)=\sum_{n=0}^{\infty}a_n\cos\big(\frac{\pi+2n\pi}{4})x\big)e^{-(\frac{\pi+2n\pi}{4})^2 t}$.结合初值条件得:
        解得\begin{equation*}
            a_n=\begin{cases}
                    4,& n=2\\
                    0.&n\neq 2
            \end{cases}
            \end{equation*}
        所以$u(x,t)=4[\cos(\frac{5\pi x}{4})]e^{-(\frac{5\pi}{4})^2}$.

    \end{enumerate}
\end{document}
