%! Author = Carl
%! Date = 2024/3/3

% Preamble
\documentclass[11pt]{article}

% Packages
\usepackage[T1]{fontenc}% optional T1 font encoding
\usepackage{graphicx}
\usepackage{color}
\usepackage{cite}
%\usepackage{tgpagella}
\usepackage{libertine}
\usepackage{subfigure}
\usepackage{amsmath}
\usepackage{amsthm}
\usepackage{ctex}
\usepackage{geometry}
% Document
\geometry{a4paper,left=2cm,right=2cm,top=1cm,bottom=3cm}

\begin{document}
\title{\vspace{-2cm}HOMEWORK\\ 数理方程与特殊函数}
\author{王翎羽\quad U202213806\quad 提高2201班}
\maketitle

\section*{练习十七}
\begin{enumerate}
    \item 证明:
    \begin{enumerate}
        \item[(1)] $\dfrac{d}{dx}[xJ_{0}(x)J_{1}(x)]=x[J_{0}^2(x)-J_{1}^{2}(x)]$;
        \item[(2)] $\displaystyle\int x^2 J_{1}(x)dx=2xJ_1(x)-x^2 J_{0}(x)+c$;
        \item[(3)] $J_{2}(x)-J_{0}(x)=2J_{0}^{''}(x)$;
        \item[(4)] $\displaystyle\int x^{n}J_{0}(x)dx=x^{n}J_{1}(x)+(n-1)x^{n-1}J_0(x)-(n-1)^2 \displaystyle\int x^{n-2}J_0(x)dx$.
    \end{enumerate}
    解:
    \begin{enumerate}
        \item[(1)]\[\begin{aligned}
                        \dfrac{d}{dx}[xJ_{0}(x)J_{1}(x)]&=\dfrac{d}{dx}[xJ_{1}(x)]J_{0}(x)+\dfrac{d}{dx}[J_{0}(x)]xJ_{1}(x)\\
                                                        &=xJ_{0}^{2}(x)-xJ_{1}^{2}(x)\\
                                                        &=x[J_{0}^2(x)-J_{1}^{2}(x)]
                   \end{aligned}\]
        \item[(2)] \[\begin{aligned}
                        \displaystyle\int x^2 J_{1}(x)dx&=-\displaystyle\int x^2 d(J_{0}(x))\\
                                                        &=-(x^2 J_{0}(x)-\displaystyle\int J_{0}(x)dx^2)\\
                                                        &=-x^2 J_{0}(x)+2\displaystyle\int J_{0}(x)dx\\
                                                        &=2xJ_1(x)-x^2 J_{0}(x)+c
                   \end{aligned}\]
        \item[(3)]\[J_{2}(x)-J_{0}(x)=-2J_{1}^{'}(x)=2J_{0}^{''}(x)\]
        \item[(4)]\[\begin{aligned}
                        \displaystyle\int x^{n}J_{0}(x)dx&=\displaystyle\int x^{n-1}d[xJ_{1}(x)]\\
                                                        &=x^{n}J_{1}(x)-(n-1)\displaystyle\int x^{n-1}J_{0}(x)dx\\
                                                        &=x^{n}J_{1}(x)-(n-1)\big(\displaystyle\int x^{n-2}d[xJ_{1}(x)]\big)\\
                                                        &=x^{n}J_{1}(x)+(n-1)x^{n-1}J_0(x)-(n-1)^2 \displaystyle\int x^{n-2}J_0(x)dx
                   \end{aligned}\]
    \end{enumerate}

    \item 计算积分:
    \begin{enumerate}
        \item[(1)] $\displaystyle\int_{0}^{3}(3-x)J_0(\dfrac{\mu_{2}^{(0)}}{3}x)dx$
        \item[(2)] $\displaystyle\int_{0}^{R}r(R^2 -r^2)J_{0}(\dfrac{\mu_{m}^{(0)}}{R}r)dr$.
    \end{enumerate}
    解:
    \begin{enumerate}
        \item[(1)]  设 $ u = \frac{\mu_{2}^{(0)}}{3} x $,则 $ x = \frac{3}{\mu_{2}^{(0)}} u $ 和 $ dx = \frac{3}{\mu_{2}^{(0)}} du $
                    代入原积分,得到:
                    \[
                    \int_0^3 (3 - x) J_0\left(\frac{\mu_{2}^{(0)}}{3} x\right) dx = \int_0^{\mu_{2}^{(0)}} \left(3 - \frac{3}{\mu_{2}^{(0)}} u\right) J_0(u) \frac{3}{\mu_{2}^{(0)}} du.
                    \]
                    简化后得到:
                    \[
                    \frac{9}{\mu_{2}^{(0)}} \int_0^{\mu_{2}^{(0)}} \left(1 - \frac{u}{\mu_{2}^{(0)}}\right) J_0(u) du=-\frac{9}{\mu_{2}^{(0)}}J_{1}(\mu_{2}^{(0)}).
                    \]

        \item[(2)]
                    设 \( u = \frac{\mu_{m}^{(0)}}{R} r \),则 \( r = \frac{R}{\mu_{m}^{(0)}} u \) 和 \( dr = \frac{R}{\mu_{m}^{(0)}} du \)。

                    代入原积分,得到:
                    \[
                    \int_0^R r (R^2 - r^2) J_0\left(\frac{\mu_{m}^{(0)}}{R} r\right) dr = \int_0^{\mu_{m}^{(0)}} \frac{R}{\mu_{m}^{(0)}} u \left(R^2 - \left(\frac{R}{\mu_{m}^{(0)}} u\right)^2\right) J_0(u) \frac{R}{\mu_{m}^{(0)}} du.
                    \]
                    简化后得到:
                    \[
                    \frac{R^3}{{\mu_{m}^{(0)}}^2} \int_0^{\mu_{m}^{(0)}} u \left(1 - \frac{u^2}{{\mu_{m}^{(0)}}^2}\right) J_0(u) du=\frac{2R^4 J_2(\mu_{m}^{(0)})}{\mu_{m}^{(0)}^{2}}.
                    \]
    \end{enumerate}


\end{enumerate}


\end{document}
