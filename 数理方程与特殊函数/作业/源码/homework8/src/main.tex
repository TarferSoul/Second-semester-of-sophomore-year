%! Author = Carl
%! Date = 2024/3/3

% Preamble
\documentclass[11pt]{article}

% Packages
\usepackage[T1]{fontenc}% optional T1 font encoding
\usepackage{graphicx}
\usepackage{color}
\usepackage{cite}
%\usepackage{tgpagella}
\usepackage{libertine}
\usepackage{subfigure}
\usepackage{amsmath}
\usepackage{amsthm}
\usepackage{ctex}
\usepackage{geometry}
\usepackage{stmaryrd}
% Document
\geometry{a4paper,left=2cm,right=2cm,top=1cm,bottom=3cm}

\begin{document}
\title{\vspace{-2cm}HOMEWORK\\ 数理方程与特殊函数}
\author{王翎羽\quad U202213806\quad 提高2201班}
\maketitle

\section*{练习十五}
\begin{enumerate}
    \item[2.] 在下半平面$y<0$内求解拉普拉斯方程的边值问题:
     \begin{equation*}
        \left\{
         \begin{array}{lr}
         u_{xx}+u_{yy}=0,  -\infty<x<+\infty, y<0,\\u|_{y=0}=f(x).
         \end{array}
        \right.
     \end{equation*}

    解:由\[u(M_0)=-\displaystyle\int_{C}f(x,y)\dfrac{\partial G}{\partial n}dS\]
    和\[G(M,M_0)=\dfrac{1}{2\pi}\big(\ln\dfrac{1}{r_{MM_0}}-\ln\dfrac{1}{r_{MM_1}}\big)\]
    由于外法向是沿Oy轴向上,所以有\[\dfrac{\partial G}{\partial n}|_{y=0}=\dfrac{\partial G}{\partial y}|_{y=0}\]
    求导可得:\[u(M_0)=\dfrac{1}{\pi}\displaystyle\int_{-\infty}^{+\infty}\dfrac{f(x)y_0 dx}{(x-x_0)^2 +y_0 ^2}\]

    \item[3.] 设$A$为常数,分别用分离变量法和格林法求解如下定解问题:
       \begin{equation*}
    \left\{
     \begin{array}{lr}
     u_{rr}+\dfrac{1}{r}u_r+\dfrac{1}{r^2}u_{\theta\theta}=0,  0<r<1,\\u(1,\theta)=A\cos\theta (-\pi<\theta\leq \pi).
     \end{array}
    \right.
    \end{equation*}
        解: \subitem(a) 分离变量法:\\
        设$u(r,\theta)=R(r)\Phi(\theta)$, 则有$R^{''}+\dfrac{1}{r}R^{'}\Phi+\dfrac{1}{r^2}R\Phi^{''}=0$.
        \\得到$r^{2}R^{''}+rR^{'}=\lambda R$, $\Phi^{''}+\lambda \Phi=0$.当$\lambda=0$时,易得$\Phi_0(\theta)=B_0$.
        \\当$\lambda>0$,$\Phi(\theta)=C_n\cos\sqrt {\lambda}\theta+D_n\sin\sqrt {\lambda}\theta $得到$\lambda=n^2,(n=1,2,\cdots)$.
        \\代入得到一个欧拉方程,解得$R_n(r)=E_n r^n$.
        \\则$u(r,\theta)=B_0(C_0\ln r +D_0)+\displaystyle\sum_{n=1}^{\infty}(E_n r^n)(C_n\cos n\theta+D_0\sin n\theta) $.
        \\$u(1,\theta)=B_0 D_0+\displaystyle\sum_{n=1}^{\infty}E_n(C_n\cos n\theta+D_0\sin n\theta)=A\cos\theta$.
        \\所以$u(r,\theta)=Ar\cos\theta.$
        \subitem(b) 格林函数法:\\
        代入公式可得:\[u(M_0)=\dfrac{1}{\pi}\displaystyle\int_{-\infty}^{+\infty}\dfrac{Axy_0 dx}{(x-x_0)^2 +y_0 ^2}\]
        又有\[u(M_0)=\dfrac{1}{2\pi}\displaystyle\int_{0}^{2\pi}f(R,\theta)\dfrac{R^2-r_0^2}{R^2+r_0^2-2Rr_0 \cos \cos(\theta-\theta-0)}d\theta\]
        经过复杂的积分计算,得到\[u(r,\theta)=Ar\cos\theta\].



    \item[4.] 设$A,B$为常数,用试探法求如下定解问题的解:
       \begin{equation*}
    \left\{
     \begin{array}{lr}
     u_{rr}+\dfrac{1}{r}u_r+\dfrac{1}{r^2}u_{\theta\theta}=0,  r<a,\\u|_{r=a}=A\cos\theta + B\cos\theta,(-\pi<\theta\leq \pi).
     \end{array}
    \right.
    \end{equation*}
        解:由边界条件设$u(r,\theta)=Cr\cos\theta+Dr\sin\theta+E$
        代入原始方程得到:$E=0,A=Ca,B=Da$.又由解的唯一性可知$C,D,E$唯一存在.
    \\[8pt] 即$u(r,\theta)=\dfrac{Ar\cos\theta}{a}+\dfrac{Br\cos\theta}{a}$.


    \end{enumerate}



\end{document}